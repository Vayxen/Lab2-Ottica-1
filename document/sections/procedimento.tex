%! TEX root = ../main.tex
\documentclass[../main.tex]{subfiles}

\begin{document}

\section{Procedimento}

Il laser è stato collocato allo zero del supporto ottico. All’altra estremità sono presenti i due sensori di luminosità e movimento solidali tra loro e collegati ad un computer tramite l’interfaccia, che servirà a registrare i dati.

Sono state spente le luci della stanza e tramite le manopole poste dietro allo strumento il puntatore è stato allineato orizzontalmente al centro dello schermo posto di fronte ad esso. Una volta rimosso lo schermo, il laser è stato allineato verticalmente al centro dell'apertura del detector per evitare l'interazione con il bordo di quest'ultima.

Successivamente è stata inserita la fenditura, regolata a \qty{0.02}{\milli\meter}, a \qty{4}{\centi\meter} di distanza dal laser ed è stata selezionata l’apertura del detector di \qty{1.5}{\milli\meter}.

Dopo aver impostato la giusta sensibilità del detector, dal computer sono state avviate le misurazioni; successivamente i sensori sono stati spostati da un estremo all’altro della guida dentata in modo da registrare l'intensità luminosa nei vari punti.

Con l'apertura del detector a \qty{1.5}{\milli\meter} sono stati effettuati più set per ciascuna fenditura in modo da verificare la ripetibilità dei dati raccolti. Successivamente è stato raccolto un set di dati per ogni combinazione di fenditura e apertura del detector per analizzare l'effetto delle variabili al contorno.

% Questo procedimento è stato ripetuto per 3 set differenti, variando per ciascuno di questi l’apertura del detector (1.5,1 e 0.5)mm e le fenditure utilizzate (0.02,0.04 e 0.08).
% %* Questo procedimento è stato poi ripetuto per le altre 2 fenditure (di 0.04 e 0.08mm) e per diverse aperture del detector(?), ottenendo 3 set in totale.
% Questi sono:
% Set 1: 10 misurazioni, 3 per ogni fenditura 0.02mm,0.04 e 0.08. Una da 0.02mm è stata ripetuta.
% - Set 2: 4 misurazioni , 1 per 0.02mm(candela),2 per 0.04 mm (una delle quali è stata effettuata con la sensibilità della candela) e 1 per 0.08mm.
% - Set 3: 2 misurazioni  1 per 0.04mm e 1 per 0.08.

% I set sono stati plottati in un grafico dell’intensità della luce in funzione della posizione e confrontati con le simulazioni effettuate utilizzando l’analisi dei dati prima con la distanza tra i minimi e poi mediante un fit.
% Dopo aver effettuato i varie set di misure abbiamo effettuato dei set sul rapporto segnale-rumore. I set effettuati sono stati sei, in due abbiamo misurato la corrente di buoi e in 4 il rumore (togliendo le fenditure):
% - Set 1 : laser costante con la sensibilità del sole.
% - Set 2: corrente di buio con la sensibilità candela.
% - Set 3: corrente di buoi con la sensibilità della lampadina.
% - Set 4: rumore con la sensibilità della lampadina. (t) in funzione del tempo?
% - Set 5: rumore con la sensibilità candela.
% - Set 6: rumore con la sensibilità della lampadina.(s) in funzione dello spazio?
% Dopo aver effettuato queste misurazioni 

\end{document}