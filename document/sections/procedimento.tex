%! TEX root = ../main.tex
\documentclass[../main.tex]{subfiles}

\begin{document}

\section{Procedimento}

Il laser è stato collocato allo zero del supporto ottico. All’altra estremità, in posizione \qty{102.50+-0.05}{\cm}, sono presenti i due sensori di luminosità e movimento solidali tra loro e collegati ad un computer tramite l’interfaccia, che servirà a registrare i dati.

Sono state spente le luci della stanza e tramite le manopole poste dietro al laser il puntatore è stato allineato orizzontalmente al centro dello schermo posto di fronte ad esso. Una volta rimosso lo schermo, il laser è stato allineato verticalmente al centro dell'apertura del detector per evitare l'interazione con il bordo di quest'ultima.

Successivamente è stata inserita la fenditura, regolata a \qty{0.02}{\milli\meter}, in posizione \qty{4.00+-0.05}{\cm} ottenendo così una distanza fenditura-detector pari a $L = \qty{98.5+-0.1}{\cm}$

Prima di effettuare le misurazioni di ogni set è stata selezionata l'apertura del detector ed è stato impostato il giusto fondoscala. Dopo aver posto i sensori ad un estremo della guida dentata, sono state avviate le misurazioni ed sono stati mossi i sensori fino alla parte opposta della guida e indietro fino a tornare al punto di partenza. Così facendo si vuole scoprire un'eventuale dipendenza delle misure dal tempo.

Con l'apertura del detector a \qty{1.5}{\milli\meter} sono stati effettuati più set per ciascuna fenditura in modo da verificare la ripetibilità dei dati raccolti. Successivamente è stata variata l'apertura del sensore per analizzare l'effetto delle variabili al contorno.

Sono anche state raccolte delle misure di intensità con il sensore otturato in modo da stimare la corrente di buio per i fondoscala utilizzati, ottenendo una corrente di buio pari a \num{0.030+-0.006} per il fondoscala \textit{lampadina} e \num{0.017+-0.005} per il fondoscala \textit{candela}.

Avendo preso le misure a partire da un'estremità della guida dentata la curva non ha il picco centrato in $y = 0$, quindi si è proceduto con il traslare la posizione delle misure fino a centrare il picco di ciascuna curva.
% L'errore attribuito a questa operazione è di $\qty{1}{\mm}$. %? review errore specifico per ogni fenditura?

A partire dai dati raccolti con apertura del sensore pari a $\qty{1.5}{\mm}$
è stata stimata la dimensione della fenditura utilizzando due metodi

\begin{enumerate}
    \item La posizione dei minimi ricavata graficamente che permette di ottenere la dimensione della fenditura utilizzando l'\autoref{eq:y=0 values}
    \item Il fit tramite l'\autoref{eq:fit}, in cui è stato inserito un parametro $c$ che permette di traslare la curva verticalmente in modo da tenere in considerazione la presenza di un offset delle misure di intensità, dovuto al rumore di fondo ed alla corrente di buio (\autoref{sec:rumore}).
\end{enumerate}

\begin{equation} \label{eq:fit}
    I_{I_{0}, a, c}(y) = I_{0} \; \sinc^{2} \left( \frac{\pi a}{\lambda} \cdot \frac{y}{L} \right) + c
\end{equation}

Infine le misure effettuate variando l'apertura del sensore sono state confrontate, per fare ciò l'intensità della misure è stata scalata in modo che l'intensità massima fosse pari a $1$, questo consente di confrontare più facilmente la larghezza del picco centrale. L'effetto atteso è una riduzione dell'intensità luminosa ed un conseguente restringimento del picco centrale derivante dalla diminuzione della superficie investita dalla luce.

\end{document}