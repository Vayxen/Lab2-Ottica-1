%! TEX root = ../main.tex
\documentclass[../main.tex]{subfiles}

\begin{document}

\section{Procedimento}

%?Prima di effettuare le misure sulla figura di diffrazione, gli strumenti sono stati sistemati come segue: il laser e la fenditura scelta (inizialmente di 0.02mm) sono stati collocati a una distanza di 3cm, mentre l'apertura del detector è stata impostata ad 1.5mm.
%Successivamente, dopo aver spento le luci e acceso il laser, è stata registrata la figura di diffrazione ottenuta tramite il software dello strumento.
%Questo procedimento è stato poi ripetuto per le altre 2 fenditure (di 0.04 e 0.08mm), ottenendo per ciascuna fenditura utilizzata 3 set in totale
%todo: (parlare su una riga a parte della parte dove si fissa una fenditura e si varia l'apertura del detector)



L’esperienza consiste nel misurare la figura di diffrazione ottenuta, cambiando ogni volta la fenditura utilizzata, %? ricavare le rispettive ampiezze dall’analisi dei dati
(consiste nel determinare l’intensità luminosa in funzione della posizione).
Sistemato il laser all’estremo del supporto ottico sono state spente le luci della stanza e %prima di procedere con l’esperimento è stato necessario assicurarsi che i raggi  fossero allineati perpendicolarmente con la fenditura di \qty{150}{\micro\meter} del detector per ottenere una lettura massima d’intensità al centro del disco arioso. Il detector è stato posto sopra in traslatore lineare. Lo schermo è posto ad una distanza L dalla fenditura, questa distanza non può essere troppo grande altrimenti avremo una dispersione di energia. 
La tensione viene misurata in uscita del fotodiodo con uno oscilloscopio. Le misure vengono effettuate quando la differenza tra massimi e minimi sarà molto piccola. Il segnale può andare a saturazione quando supera la sensibilità del detector.


\end{document}