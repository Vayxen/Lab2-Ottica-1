%! TEX root = ../main.tex
\documentclass[../main.tex]{subfiles}

\begin{document}

\section{Procedimento}

Il laser è stato collocato allo zero del supporto ottico. All’altra estremità, in posizione \qty{102.50+-0.05}{\cm}, sono presenti i due sensori di luminosità e movimento solidali tra loro e collegati ad un computer tramite l’interfaccia, che servirà a registrare i dati.

Sono state spente le luci della stanza e tramite le manopole poste dietro al laser il puntatore è stato allineato orizzontalmente al centro dello schermo posto di fronte ad esso. Una volta rimosso lo schermo, il laser è stato allineato verticalmente al centro dell'apertura del detector per evitare l'interazione con il bordo di quest'ultima.

Successivamente è stata inserita la fenditura, regolata a \qty{0.02}{\milli\meter}, in posizione \qty{4.00+-0.05}{\cm} ottenendo così una distanza fenditura-detector pari a $L = \qty{98.5+-0.1}{\cm}$

Prima di effettuare le misurazioni di ogni set è stata selezionata l'apertura del detector ed è stato impostato il giusto fondoscala. Dopo aver posto i sensori ad un estremo della guida dentata, sono state avviate le misurazioni ed sono stati mossi i sensori fino alla parte opposta della guida e indietro fino a tornare al punto di partenza. Così facendo si vuole scoprire un'eventuale dipendenza delle misure dal tempo.

Con l'apertura del detector a \qty{1.5}{\milli\meter} sono stati effettuati più set per ciascuna fenditura in modo da verificare la ripetibilità dei dati raccolti. Successivamente è stata variata l'apertura del sensore per analizzare l'effetto delle variabili al contorno.

\end{document}