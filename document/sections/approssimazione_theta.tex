%! TEX root = ../main.tex
\documentclass[../main.tex]{subfiles}

\begin{document}

\section{Approssimazione angolo $\theta$} \label{sec:approssimazione theta}

La legge che descrive l'intensità della luce su un punto dello schermo a distanza $y$ dal centro è l'\autoref{eq:I su theta}.

\begin{equation} \label{eq:I su theta}
    I(\theta) = I_{0} \; \sinc^{2} \left( \frac{\pi a}{\lambda} \cdot \sin(\theta)  \right)
\end{equation}

in cui $\theta$ è l'angolo formato in corrispondenza della fenditura tra la retta perpendicolare allo schermo, passante per il suo centro, e quella passante per il punto dello schermo preso in analisi.

Dato che la distanza tra la fenditura e lo schermo $L \gg y$ è possibile applicare l'approssimazione in \autoref{eq:approx theta}.

\begin{equation} \label{eq:approx theta}
    \sin(\theta) \approx \theta = \arctan\left( \frac{y}{L} \right) \approx \frac{y}{L} 
\end{equation}

Si giunge quindi all'\autoref{eq:I su y} utilizzata per il fit.

\end{document}