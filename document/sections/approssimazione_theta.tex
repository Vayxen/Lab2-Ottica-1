%! TEX root = ../main.tex
\documentclass[../main.tex]{subfiles}

\begin{document}

\section{Approssimazione angolo $\theta$} \label{sec:approssimazione theta}

La legge che descrive l'intensità della luce su un punto dello schermo a distanza $y$ dal centro è l'\autoref{eq:I su theta}.

\begin{equation} \label{eq:I su theta}
    I(\theta) = I_{0} \; \sinc^{2} \left( \frac{\pi a}{\lambda} \cdot \sin(\theta)  \right)
\end{equation}

in cui $\theta$ è l'angolo formato in corrispondenza della fenditura tra la retta perpendicolare allo schermo, passante per il suo centro, e quella passante per il punto dello schermo preso in analisi.

La distanza massima dal centro dello schermo raggiunta dal sensore è di $y = \qty{10}{\cm}$ e la distanza fenditura-sensore è pari a $L = \qty{98.5}{\cm}$ il che implica che il massimo angolo $\theta$ raggiunto è pari a 

\begin{equation*}
    \theta_{max} = \arctan\left( \frac{y}{L} \right) \approx \qty{5.8}{\degree}
\end{equation*}

valore per il quale è possibile applicare l'approssimazione in \autoref{eq:approx theta}.

\begin{equation} \label{eq:approx theta}
    \sin(\theta) \approx \theta = \arctan\left( \frac{y}{L} \right) \approx \frac{y}{L} 
\end{equation}

Quindi in \autoref{eq:I su theta} è possibile sostituire a $\sin(\theta)$ il rapporto $\dfrac{y}{L}$ giungendo all'\autoref{eq:I su y}.

\end{document}