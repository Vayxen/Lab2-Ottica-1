%! TEX root = ../main.tex
\documentclass[../main.tex]{subfiles}

\begin{document}

\section{Risultati}

Avendo preso le misure a partire da un'estremità della guida dentata la curva non ha il picco centrato in $y = 0$, quindi si è proceduto con il traslare la posizione delle misure fino a centrare il picco di ciascuna curva.
% L'errore attribuito a questa operazione è di $\qty{1}{\mm}$. %? review errore specifico per ogni fenditura?

A partire dai dati raccolti con apertura del sensore pari a $\qty{1.5}{\mm}$ 
è stata stimata la dimensione della fenditura utilizzando due metodi

\begin{enumerate}
    \item La posizione dei minimi ricavata graficamente che permette di ottenere la dimensione della fenditura utilizzando l'\autoref{eq:y=0 values}
    \item Il fit tramite l'\autoref{eq:fit}, in cui è stato inserito un parametro $c$ che permette di traslare la curva verticalmente in modo da tenere in considerazione la presenza di rumore.
\end{enumerate}

\begin{equation} \label{eq:fit}
    I(I_{0}, a, c) = I_{0} \; \sinc^{2} \left( \frac{\pi a}{\lambda} \cdot \frac{y}{L} \right) + c
\end{equation}

\subfile{0.02.tex}

\newpage~\newpage~\newpage

\subfile{0.04.tex}

\newpage~\newpage~\newpage

\subfile{0.08.tex}

\end{document}