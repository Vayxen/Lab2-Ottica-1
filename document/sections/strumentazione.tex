%! TEX root = ../main.tex
\documentclass[../main.tex]{subfiles}

\begin{document}

\section{Strumentazione}

\begin{itemize}
    \item \textbf{Laser} con lunghezza d'onda $\lambda = \qty{650}{\nano\meter}$
    \item \textbf{Fenditura} di larghezza variabile da \num{0.02}, \num{0.04} e \qty{0.08}{\milli\meter}
    \item \textbf{Guida con riga} di lunghezza pari a \qty{1.2}{\meter} e risoluzione \qty{1}{\milli\meter}, su cui montare i vari strumenti
    \item \textbf{Schermo} utile per centrare il laser orizzontalmente
    \item \textbf{\textit{Light sensor}} in grado di campionare l'intensità luminosa con tre diverse scale e dotato di fenditura variabile \num{0.5}, \num{1} e \qty{1.5}{\milli\meter}
    \item \textbf{\textit{Rotary motion sensor}} capace di misurare la rotazione relativa al punto di avvio della misurazione con una risoluzione di \qty{0.09}{\degree}. Il fattore di conversione lineare utilizzando la guida dentata, indicato nel manuale, è circa \qty{0.0126}{\meter/\radian}
    \item \textbf{Guida dentata} lunga \qty{21}{\centi\meter} su cui è montato il sensore di rotazione
    \item \textbf{Interfaccia} per collegare il sensore ad un computer
\end{itemize}

\subsection{Software}

\begin{itemize}
    \item \textbf{\href{https://www.pasco.com/downloads/capstone}{\underline{Pasco Capstone}}} per controllare l'interfaccia
\end{itemize}

\end{document}