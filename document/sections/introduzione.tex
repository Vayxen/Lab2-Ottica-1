%! TEX root = ../main.tex
\documentclass[../main.tex]{subfiles}

\begin{document}

\section{Introduzione}

Questo esperimento vuole analizzare il carattere ondulatorio della luce tramite il fenomeno della \textbf{diffrazione}, causata dal passaggio del fascio di luce per una fenditura di dimensioni $d$ comparabili alla sua lunghezza d'onda $\lambda$.

%\noindent Le leggi che esprimono la larghezza $a$ della fenditura, l'intensità $I$ in funzione della distanza dal picco centrale $y_0$ e la sua larghezza sulla figura di interferenza, sono indicate qui di seguito insieme alla spiegazione di ciascun parametro:

% \begin{align}
%     a &= \frac{m\lambda d}{y_m} \\
%     I(\theta) &= I_{0} \; \text{sinc}^{2}\left(\frac{\pi a \sin{\theta}}{\lambda}\right) \\
%     %!rimpiazza questo commento con l'equazione
% \end{align}

\noindent Nell'esperimento si cercherà di ({\color{red}ottenere? derivare? stimare?}) sperimentalmente le dimensioni della fenditura utilizzata (confrontando il risultato con le dimensioni nominali) e di ricavare la larghezza del picco centrale. %mi manca qualcosa penso



\end{document}