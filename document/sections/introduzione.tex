%! TEX root = ../main.tex
\documentclass[../main.tex]{subfiles}

\begin{document}

\section{Introduzione}

Questo esperimento vuole rilevare il carattere ondulatorio della luce tramite il fenomeno della \textbf{diffrazione}, causata dal passaggio del fascio di luce per una fenditura di dimensioni $a$ comparabili alla sua lunghezza d'onda $\lambda$. Per far ciò verrà analizzata la figura d'interferenza formata su uno schermo a distanza $L$ dalla fenditura (\autoref{fig:fenditura}).

\begin{figure}[ht!]
    \centering
    \includefig{0.5\textwidth}{fenditura}
    \caption{Illustrazione dell'apparato strumentale}
    \label{fig:fenditura}
\end{figure}

La legge che descrive l'intensità della luce su un punto dello schermo a distanza $y$ dal centro è l'\autoref{eq:I su y}.

% todo: non so se inserire questa parte perché mi sempra troppo uno spiegone
% La legge che descrive l'intensità della luce su un punto dello schermo a distanza $y$ dal centro è l'\autoref{eq:I su theta}.

% \begin{equation} \label{eq:I su theta}
%     I(\theta) = I_{0} \; \sinc^{2} \left( \frac{\pi a}{\lambda} \cdot \sin(\theta)  \right)
% \end{equation}

% in cui $\theta$ è l'angolo formato in corrispondenza della fenditura tra la retta perpendicolare allo schermo, passante per il suo centro, e quella passante per il punto dello schermo preso in analisi.

% Dato che la distanza tra la fenditura e lo schermo $L \gg y$ è possibile applicare l'approssimazione in \autoref{eq:approx theta}.

% \begin{equation} \label{eq:approx theta}
%     \sin(\theta) \approx \theta = \arctan\left( \frac{y}{L} \right) \approx \frac{y}{L} 
% \end{equation}

% Si giunge quindi all'\autoref{eq:I su y} utilizzata per il fit.

\begin{equation} \label{eq:I su y}
    I(y) = I_{0} \; \sinc^{2} \left( \frac{\pi a}{\lambda} \cdot \frac{y}{L}  \right)
\end{equation}

Per trovare i punti di minimo basta porre $\dfrac{a y}{\lambda L} \in \mathbb{Z}_{\setminus \{0\}}$ che può essere scritto come in \autoref{eq:y=0 values}.

\begin{equation} \label{eq:y=0 values}
    y \in \left\{m\frac{\lambda L}{a}: m \in \mathbb{Z}_{\setminus \{0\}} \right\}
\end{equation}

% Nell'esperimento si cercherà di ({\color{red}ottenere? derivare? stimare?}) sperimentalmente le dimensioni della fenditura utilizzata (confrontando il risultato con le dimensioni nominali) e di ricavare la larghezza del picco centrale. %mi manca qualcosa penso

\end{document}