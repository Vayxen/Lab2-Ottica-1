%! TEX root = ../main.tex
\documentclass[../main.tex]{subfiles}

\begin{document}

\section{Introduzione}

Questo esperimento vuole rilevare il carattere ondulatorio della luce tramite il fenomeno della \textbf{diffrazione}, causata dal passaggio del fascio di luce per una fenditura di dimensioni $a$ comparabili alla sua lunghezza d'onda $\lambda$. Per far ciò verrà analizzata la figura d'interferenza formata su uno schermo a distanza $L$ dalla fenditura (\autoref{fig:fenditura}).

\begin{figure}[ht!]
    \centering
    \includefig{0.5\textwidth}{fenditura}
    \caption{Illustrazione dell'apparato strumentale}
    \label{fig:fenditura}
\end{figure}

La legge che descrive l'intensità della luce su un punto dello schermo a distanza $y$ dal centro è l'\autoref{eq:I su y} come ricavato in \autoref{sec:approssimazione theta}.

\begin{equation} \label{eq:I su y}
    I(y) = I_{0} \; \sinc^{2} \left( \frac{\pi a}{\lambda} \cdot \frac{y}{L}  \right)
\end{equation}

Per trovare i punti di minimo basta porre $\dfrac{a y}{\lambda L} \in \mathbb{Z} \setminus \{0\}$ ovvero

\begin{equation} \label{eq:y=0 values}
    y \in \left\{m\frac{\lambda L}{a}: m \in \mathbb{Z} \setminus \{0\} \right\}
\end{equation}

\end{document}