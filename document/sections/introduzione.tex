%! TEX root = ../main.tex
\documentclass[../main.tex]{subfiles}

\begin{document}

\section{Introduzione}

Questo esperimento vuole rilevare il carattere ondulatorio della luce tramite il fenomeno della \textbf{diffrazione}, causata dal passaggio del fascio di luce per una fenditura di dimensioni $a$ comparabili alla sua lunghezza d'onda $\lambda$, analizzando la figura d'interferenza formata su uno schermo a distanza $L$ (\autoref{fig:fenditura}).

\begin{figure}[ht!]
    \centering
    \includefig{0.5\textwidth}{fenditura}
    \caption{Illustrazione dell'apparato strumentale}
    \label{fig:fenditura}
\end{figure}

La legge che descrive l'intensità della luce su un punto dello schermo è l'\autoref{eq:I su theta}

\begin{equation} \label{eq:I su theta}
    I(\theta) = I_{0} \; \sinc^{2} \left( \frac{\pi a}{\lambda} \cdot \sin(\theta)  \right)
\end{equation}

in cui $\theta$ è l'angolo formato in corrispondenza della fenditura tra la retta perpendicolare allo schermo, passante per il suo centro, e quella passante per il punto dello schermo preso in analisi.

Dato che la distanza tra la fenditura e lo schermo $L \gg y$ è possibile applicare l'approssimazione in \autoref{eq:approx theta}

\begin{equation} \label{eq:approx theta}
    \sin(\theta) \approx \theta = \arctan\left( \frac{y}{L} \right) \approx \frac{y}{L} 
\end{equation}

dove $y$ indica la distanza del punto in analisi dal centro dello schermo.

Si giunge quindi all'\autoref{eq:I su x} utilizzata per il fit.

\begin{equation} \label{eq:I su x}
    I(y) = I_{0} \; \sinc^{2} \left( \frac{\pi a}{\lambda} \cdot \frac{y}{L}  \right)
\end{equation}

Per trovare i punti di minimo basta porre $\dfrac{a y}{\lambda L} \in \mathbb{Z}_{\setminus \{0\}}$ ovvero

\begin{equation} \label{eq:y=0 values}
    y \in \left\{m\frac{\lambda L}{a}: m \in \mathbb{Z}_{\setminus \{0\}} \right\}
\end{equation}

%Le leggi che esprimono la larghezza $a$ della fenditura, l'intensità $I$ in funzione della distanza dal picco centrale $y_0$ e la sua larghezza sulla figura di interferenza, sono indicate qui di seguito insieme alla spiegazione di ciascun parametro:

% \begin{align}
%     a &= \frac{m\lambda d}{y_m} \\
%     I(\theta) &= I_{0} \; \text{sinc}^{2}\left(\frac{\pi a \sin{\theta}}{\lambda}\right) \\
%     %!rimpiazza questo commento con l'equazione
% \end{align}

% Nell'esperimento si cercherà di ({\color{red}ottenere? derivare? stimare?}) sperimentalmente le dimensioni della fenditura utilizzata (confrontando il risultato con le dimensioni nominali) e di ricavare la larghezza del picco centrale. %mi manca qualcosa penso



\end{document}