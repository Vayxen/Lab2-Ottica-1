%! TEX root = ../main.tex
\documentclass[../main.tex]{subfiles}

\begin{document}

\section{Conclusioni}

Tramite questo esperimento è stato possibile indagare sulla natura ondulatoria della luce.

Per la fenditura da \qty{0.02}{\mm} è stata formulata l'ipotesi della presenza di un difetto nella stessa che causa fenomeni di interferenza, i quali si sovrappongono alla figura di diffrazione producendo quelle fluttuazioni di intensità periodiche visibili in \autoref{fig:single scatter 0.02}.

Riguardo l'asimmetria delle curve, non si è stati in grado di individuarne la causa, in quanto nonostante la principale fonte di rumore fosse sulla destra della guida dentata, le intensità maggiori si registrano alla sinistra di essa. %* tiro a caso e ipotizzo che magari la "destra della guida" è esattamente la sinistra del grafico, magari hai invertito il ragionamento?

Variando l'apertura del sensore non si nota alcuna variazione consistente nella larghezza del picco centrale.

Infine è possibile notare come tutti i valori delle fenditure, siano esse ottenute graficamente o tramite fit, risultano essere maggiori del valore nominale della fenditura corrispondente. L'ipotesi è che sia stato utilizzato un valore errato % della distanza fenditura-sensore o
del fattore di conversione per trasformare le misure di rotazione (in \si{\radian}) ottenute tramite il \textit{rotary motion sensor} in misure di traslazione (in \si{\m}).

Imponendo il valore nominale delle varie fenditure all'interno dell'\autoref{eq:y=0 values} si ottiene che % la distanza dovrebbe essere di \qty{92.5+-1.5}{\cm} oppure in alternativa
il fattore di conversione dovrebbe essere \num{0.01335+-0.00015}. Non essendoci prove a favore di questa tesi essa rimane unicamente un'ipotesi.
% nessuna delle due tesi esse restano semplicemente delle ipotesi.

\end{document}